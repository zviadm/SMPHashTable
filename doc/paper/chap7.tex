\section{Future Work}
\label{chap:futurework}

\cphash{} demonstrates that by using computation migration, message passing and careful memory management techniques it is
possible to create a scalable hash table implementation for the multi-core NUMA architecture machines. However, there is still
room for improvement of \cphash{}'s implementation. 

\subsection{Dynamic Adjusting of the Server Threads}

One issue with the current design is that a fixed number of cores must to be dedicated to run the server threads. 
A better approach would be to have an algorithm that would dynamically decide on how many cores to use for the server threads, 
depending on the workload. Such dynamic adjustment of the server threads would make it possible to use less power and CPU resources
when the workload is small. Saving power is essential for any data center due to reduced cost. Using less CPU resources
would make it possible to run other services on the same machine when there is not much load on the hash table. 

Dynamic adjustment of the server threads could also provide higher performance. 
If the CPU resources needed by the client threads to generate the queries is less than the resources needed by the server threads 
to complete the queries, then it is better to dedicate more cores to run the server threads than to the client threads. 
On the other hand, if the client threads need more CPU resources to generate the queries, it is better to dedicate fewer cores to run
the server threads and use more cores for the client threads. 

We have not tried implementing dynamic adjustment of server threads due to time constraint reasons; however, we tried
a different approach to avoid wasting the CPU resources. We tried utilizing CPUs with Intel's HyperThreading technology to run 
the server and the client threads on two separate logical cores of the same physical core. The problem we discovered with this approach is that 
since the logical cores share the L2 cache, the client thread can easily pollute the whole cache thus nullifying most of the benefits of the \cphash{}
design. This is especially true for an application such as \cpserver{}, since the connection buffers themselves can take most of the space in the cache.

\subsection{Message Passing}

The scalability and performance of message passing is important for \cphash{}. The current implementation is simple and does not use
any hardware specific operations to further speedup the communication between the server and the client threads. One possible improvement is to
forcefully flush the caches when the buffer is full, this way the overhead of sending the message would shift more from the receiver towards the sender. 
Such an approach could be beneficial to decrease the time spent on reading the message buffers in server threads. Such an approach could potential provide
more scalable message passing implementation. 

Another idea to improve the current message passing implementation is to change the design of the circular buffer to eliminate the read and write indexes
to further decrease average cache misses spent on message passing per each query. 

