\chapter{Conclusion}

In this thesis we introduced \cphash{} $-$ a scalable fixed size hash table implementation that supports eviction
using an LRU list, and \cpserver{} $-$ a scalable in memory key/value cache server implementation that uses \cphash{}
as its hash table. 
Experiments on a 48 core machine showed that on a small hash table \cphash{} has 3 times higher throughput than a hash
table implemented using scalable fine-grained locks. On a large hash table \cphash{} had 2 times higher throughput than
a hash table implemented using scalable locks. 
\cpserver{} achieved 1.2 to 1.7 times higher throughput than a key/value cache server 
that uses a hash table with scalable fine-grained locks, and 1.5 to 2.6 times higher throughput than \memcached{}.

The improved performance is due to the reduced number of cache misses per operation. The number of cache coherency misses
is reduced due to caching common partition data that are modified, which in \cphash{} is the head of the LRU list. 
For small hash tables, the number of cache capacity misses is reduced, since \cphash{} avoids the data duplication in local hardware
caches by transferring the computation between the cores instead of the data.

