\chapter{Memory Management}

To implement a non-intrusive hash table, in addition to storing keys and pointers to the values, the actual value data
needs to be stored. In order to store arbitrary length values, \cphash{} needs the ability to allocate space in memory 
when inserting an element and free it when \cphash{} evicts an element from the hash table. We also need to decide 
which thread (server or client) should be responsible for data allocation and which thread should be responsible 
for copying the data into the allocated space. Freeing values is complicated by the fact that each value can be in use 
in multiple client threads; thus, we need some way of determining when it is actually safe to free and reuse 
the space used by a data value.

Section 4.1 discusses allocation strategies and provides details on our actual implementation. Section 4.2 
discuses strategies for freeing and deallocation. In Section 4.3 we discuss the alternative strategy of reference counting.

\section{Allocation}

The best place to do space allocation is in the server thread since each server is responsible for a single partition and 
implementing space allocation would be as simple as implementing a space allocator for a single-threaded hash table. 
However, performing the actual data copying in the server thread would be a bad idea since for large values it would 
wipe-out the local hardware cache of the server core. Thus, in \cphash{} the space allocation is done in the server 
thread and the actual data copying is performed in the client thread. To perform an \texttt{Insert} operation, the client sends the \texttt{key} and 
the \texttt{size} of the value to the server. The server allocates \texttt{size} amount of space in memory and returns 
the pointer to the allocated memory to the client. The allocated space is marked as NOT READY and will not be used until it is marked as READY. 
The client receives the pointer, copies the data to the location pointed by the given pointer, and marks that space as READY. 
The current \cphash{} implementation this marking is done using atomic operations. In Section 4.3 we discuss an alternative 
to it using message passing.

There are many different ways to perform data allocation in the server thread. The simplest way is to use C standard 
\texttt{malloc}/\texttt{free} operations. However, in a heavily multi-threaded environment the libc standard \texttt{malloc} performs poorly. 
A better alternative could be to use memory allocators designed for multi-threaded programs such as streamflow \cite{streamflow}, 
or tcmalloc \cite{tcmalloc} or any other multi-threaded allocator. However, since in \cphash{} the total space is split equally 
between partitions, we decided to just pre-allocate all of the available space for each partition and then use the standard
single-threaded binning allocator \cite{binallocator} inside that pre-allocated space. This way the server threads will never have to communicate when 
allocating or freeing space. 

\section{Deallocation/Freeing}

When the server thread evicts or deletes an element from the hash table, the space allocated for this value must to be 
freed so that it can be reused for new elements. It would be incorrect for server thread to just free the allocated space when it evicts or deletes 
the element. The problem is that if a client requests a \texttt{Lookup} on some element X and gets the pointer to its value, 
and then the server threads evicts the element X from the hash table before the client is done processing X's value, the client will have a dangling 
pointer pointing to unallocated space, potentially causing all kinds of errors. To resolve this issue, \cphash{} counts references to the elements. 
Each element in the hash table has a reference count. Every time a client requests a \texttt{Lookup} of an element, the server thread increases the element's 
reference count. When the client is done with the item it decreases the reference count of the given element. When the reference count reaches 0, 
the space can be safely deallocated. We implemented reference counting using atomic operations.

The deallocation must be done by server threads, otherwise there would be race conditions between allocations 
and deallocations for a partition. When a client dereferences an element and its reference count becomes zero we need 
some way to schedule an element for freeing in the server thread. It is worth noting that this is, in general, a highly unlikely 
scenario, especially if clients process elements quickly, since if an element was just accessed in the hash table, it 
would become the most-recently used item thus significantly decreasing chances of its eviction before the client is done processing it.

To implement scheduling of elements for freeing, we implemented a lock-free singly-linked list using atomic operations that holds 
the list of elements that can be safely freed. Scheduled elements are freed in the server thread before the next allocation.

\section{Atomic Operations VS Message Passing}

As mentioned in previous sections we implemented the necessary synchronization for memory management using hardware-supported atomic 
operators. Another way to implement reference counting could have been using message passing. Instead of the client thread 
updating the reference count, it would send a message to the appropriate server to update the reference counter. However, in this 
case, message passing is not the best option for several reasons. Since message passing is implemented 
using shared memory and without special hardware support, sending a single message is just as expensive as a single write 
to a memory location. Also in most common cases when a client needs to update the reference count, the counter is already in 
the client's cache thus making those atomic operations fast. The message passing version could become a more viable option 
if the hardware had some special support for fast core-to-core communication.

Another alternative way to implement reference counting would be to send a single message to the server to release all pointers 
per batch. However, that would require the server thread to store all the pointers allocated during the last batch for each client. 
This would impose a significant overhead on the server's local hardware cache, especially if the batches are large. Also it would provide less 
flexibility for a client to decide when to release values.
