\chapter{Introduction}
\label{chap:intro}

Hash tables are heavily used data structures in distributed data centers and web servers. This thesis focuses
on fixed-size hash tables that support eviction of its elements using a Least Recently Used (LRU) list. Such hash tables are a good way to
implement a key/value cache. One of the best known distributed applications that uses a key/value cache is 
\memcached{} \cite{memcached}. \memcached{} is an in-memory cache for Web applications that store  
data, page rendering results, and other information that can be cached and is expensive to recalculate.

With the rapid growth of the World Wide Web and large-scale applications, more scalability is demanded from data structures. 
Although many data structures and applications are already developed with the scalability requirement in mind, 
they are usually designed for distributed systems that consist of multiple different machines. The recent emergence of multi-core 
architectures demands that we rethink scalability not just for scaling across multiple machines but also across 
multiple cores of the same machine. 

This thesis explores the use of \textit{computation migration} to increase data structure performance on 
multi-core processors. This technique can also be applied to structures other than the hash table. We chose the 
hash table to demonstrate the benefits of this idea because of its overall simplicity, ease of implementation and relevance.

\section{Motivation}

Since CPU frequencies can no longer be significantly increased due to heat and power dissipation challenges, 
processors are now becoming more powerful by having more and more separate computation cores. Thus, if we want to 
get better performance as the number of cores increases, we need to think of ways to make our applications more scalable across 
multiple computation cores. One of the first steps in this challenge is to rethink the current data structures with 
the multi-core model in mind.  

In a multi-core processor that supports shared memory each core has its own data cache to make memory accesses faster. 
However, if multiple cores are modifying and reading the same data, it is becoming more and more expensive to keep the caches 
coherent. Clearly, as the number of cores continues to increase, it will become more and more expensive to access and modify 
the same data using multiple cores. Also most of the current processors have NUMA architecture, thus localized memory accesses
are faster than random memory accesses.

In the 48-core machine used in thesis, a core can access its L1 cache in 3 cycles, its L2 cache in 14 cycles, and the shared on-chip 
L3 cache in 28 cycles. DRAM access latencies vary, from 122 cycles for a core to read from its local DRAM to 503 cycles for a 
core to read from the DRAM of the chip farthest from it on the interconnect. Since accesses from caches are much faster than 
accesses from the DRAM, reducing the number of cache misses can have large impact on overall performance of an application.


\section{This Thesis}

This thesis introduces a new hash table, which we call \cphash{}, which uses computation migration to avoid unnecessary data transfer between cores 
and increase performance and scalability by reducing total number of cache misses. Instead of all the cores accessing shared data, \cphash{} splits the 
contents of the data structure into multiple parts and assign a core to each particular part of the structure. \cphash{} uses message passing 
to pass the lookup/insert operation to the core that is assigned the data needed for that particular operation. \cphash{} assumes that computation migration will 
work well when modifiable data contents are large and the computation description is small. This thesis strives to prove this 
assumption by providing an implementation of \cphash{} that uses computation migration and demonstrating its performance gains. 

On the 48-core machine we compare the performance of \cphash{} to the performance of a standard fine grain lock implementation of a hash table and 
observe benefits due to two main reasons: Decrease in cache capacity misses (for small data sets) and decrease in cache coherency misses (for all data sets).
Cache capacity misses are reduced since \cphash{} avoids data transfers and thus data duplication in different caches. Cache coherency misses
are reduced due to caching common partition data that are modified, which in \cphash{} is the head of the LRU list. Both the lookup and the insert 
operations access and modify the head of the LRU list.

This thesis also introduces a memcached style key/value cache server, which we call \cpserver{}, which uses \cphash{} as its hash table. We compare the performance of \cpserver{} to the
performance of a key/value cache server that uses a hash table with standard fine grain locks. We also compare the performance of \cpserver{} against \memcached{}.
We observe increase in throughput in both cases due to the speedup provided by the \cphash{} hash table implementation. 

\section{Outline}

The remainder of this thesis is structured as follows. Chapter 2 describes the related work. Chapter 3 
describes the overall design and implementation of \cphash{}. Chapter 4 describes \cphash{}'s memory management 
algorithm for allocating and storing the hash table contents in memory. Chapter 5 describes design and protocol of \cpserver{}.
Chapter 6 describes the benchmarking methods and contains a detailed evaluation of the performance gains. In Chapter 7 we discuss future plans for \cphash{}. 
Finally, Chapter 8 concludes this thesis with an overall perspective and summary of the achieved results. 

