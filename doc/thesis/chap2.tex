\chapter{Related Work}

There already exists many different techniques to optimize use of caches on multi-core chips.
In this chapter we present overview of some of those methods and describe how they differ from the approach
taken in this thesis.

\section{Multi-core Cache Management}

Thread clustering~\cite{tam:threadclustering} dynamically clusters
threads with their data on to a core and its associated cache. Chen et
al.~\cite{chen-07} investigate two schedulers that attempt to schedule
threads that share a working set on the same core so that they share
the core's cache and reduce DRAM references.  Several researchers have
used page coloring to attempt to partition on-chip caches between
simultaneous executing applications~\cite{cho:micro,tam:sharedl2,lin:partitionl2,soares:pollute,zhang:pagecolor}. 
Chakraborty et al.~\cite{koushik:csp} propose computation spreading, which uses
hardware-based migration to execute chunks of code from different
threads on the same core to reduce i-cache misses.

Several researchers place OS services on particular cores and invoke them with messages.
Corey~\cite{corey:osdi08} can dedicate a core to handling a particular network device and its
associated data structures.  Mogul et al. optimize some cores for energy-efficient execution of OS code~\cite{mogul:micro}.
Suleman et al. put critical sections on fast cores~\cite{suleman:acs}.
Barrelfish~\cite{barrelfish} and fos~\cite{wentzlaff:fos} treating cores as independent nodes that
communicate using message passing. 

The methods described in this thesis focus specifically on data structures and are meant for providing techniques for scaling the 
data structures on many core processors. Flat combining~\cite{flatcombining} has the same motivation. The main idea behind flat combining
is to let a single thread gain global lock on a data structure and perform all the operations on it that all the other threads
have scheduled. This way when multiple threads are competing for the global lock only one of them has to acquire it; others can just
schedule their operation and wait for the result. This approach is somewhat similar to the approach that we take with \cphash{} in a sense that 
there is a server thread that performs all operations and there are client threads that schedule their operations. The main difference is
that in \cphash{} there are multiple dedicated server threads that perform the operations and this server threads are pinned to specific cores.
On the other hand in flat combining there is a single thread at any time that acts as a server thread, but any thread can become the
server thread.

\section{Computation Migration}

\cphash{} attempts to move computation close to data, and was inspired by 
computation migration in distributed shared memory systems such as MCRL~\cite{hsieh:sc} and
Olden~\cite{olden} and remote method invocation in parallel programming languages such as 
Cool~\cite{COOL} and Orca~\cite{orca:tocs}.

